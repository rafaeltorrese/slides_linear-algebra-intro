\section{Sistemas de Ecuaciones Lineales}
\label{sec:linear-equations}

\begin{frame}{Solución de Un Sistema de Ecuaciones Lineales}

  \begin{columns}
    \column{0.6\textwidth}
      {\centering
  \resizebox{0.9\textwidth}{!}{%
    \begin{tikzpicture}
      
      % Grid
      \draw[black!10,dashed,xstep=1cm] (-6,-6) grid(6,6);
      

      % Axes:
      % Are simply drawn using line with the `->` option to make them arrows:
      % The main labels of the axes can be places using `node`s:
      \draw [->] (-6,0) -- (6,0) node [above left]  {$x$};
      \draw [->] (0,-6) -- (0,6) node [below right] {$y$};

      \draw [name path={blue line}, blue] (-3.94,-6.84) -- (0.79,7.36) node [below right] {$L_1$};
      \draw [name path={red line}, red] (-3.94, -4.894) -- (0.789, 4.57) node [below right] {$L_2$};

      % Intersections
      \fill[name intersections={of=blue line and red line}](intersection-1) circle (3pt);
    \end{tikzpicture}
  }
  \par}

\column{0.3\textwidth}
\begin{onlyenv}<1>
  \begin{align*}
    y &= 3x + 5\\
    y &= 2x + 3\\[5mm]
    3x + 5 & = 2x + 3\\
    3x - 2x &= 3 - 5\\
    x&= \\
    y &= 
  \end{align*}
\end{onlyenv}
\begin{onlyenv}<2>
  \begin{align*}
    y &= 3x + 5\\
    y &= 2x + 3\\[5mm]
    -3x + y &= 5\\
    -2x + y &= 3
  \end{align*}
  \begin{flalign*}
    \bm{A} & =
      \begin{pmatrix}
      -3 & 1\\
      -2 & 1
    \end{pmatrix}
    \\
    \bm{x} &=%
    \begin{pmatrix}
      x \\
      y
    \end{pmatrix}
    \\
    \bm{b} &=%
    \begin{pmatrix}
      5 \\
      3
    \end{pmatrix}
  \end{flalign*}
\end{onlyenv}
  \end{columns}
\end{frame}

\begin{frame}{Tipos Especiales de Matrices}
  \begin{columns}
    \column{0.4\textwidth}
      \begin{itemize} \justifying \parskip3mm
  \item Matriz Identidad $I$.
  \item  Matriz Inversa $A^{-1}$.
  \item  Matriz Singular.
  \end{itemize}
  \column{0.6\textwidth}
  \lstinputlisting[firstline=3]{scriptsLinAlg/09_special-matrix.py}
  \end{columns}
\end{frame}

\begin{frame}{Solución de Ecuaciones con Matriz Inversa}

  \begin{theorem}\justifying
    Si $\bm{A}$ es una matriz  invertible de $n \times n$, entonces para toda matriz $\bm{b}$ de $n \times 1$, el sistema de ecuaciones $A\bm{x} = \bm{b}$ tiene exactamente una solucioń; a saber, $\bm{x} = A^{-1}\bm{b}$
  \end{theorem}
  \begin{flalign*}
    x_1 + x_2 + 2x_3 & = 9\\
    2x_1 + 4x_2 - 3x_3 & = 1\\
    3x_1 + 6x_2 - 5x_3 & = 0\\
  \end{flalign*}
\end{frame}




\begin{frame}{Sistemas Sin Solución, Una Solución, Infinitas Soluciones}

  Todo sistema de Ecuaciones lineales no tiene soluciones, tiene exactamente una solución o tiene una infinidad de soluciones.

  \begin{columns}[t]
    \column{0.3\textwidth}
      \begin{flalign*} 
  y &= 3x + 5 \\
  y &= 6x  + 7 \\
  y &= \frac{5}{2}x - 1  
\end{flalign*}

\column{0.3\textwidth}
\begin{flalign*}
  y &= \nicefrac{1}{2}x + 0.5 \\
  y &= \nicefrac{-3}{5}x + \nicefrac{8}{5}\\
  y &= \nicefrac{-4}{3}x + \nicefrac{7}{3} 
\end{flalign*}
\column{0.3\textwidth}
\begin{flalign*}
  y &= 2x + 5 \\
  y &= 2x + 5
\end{flalign*}
  \end{columns}
  
\end{frame}


\begin{frame}{Graficar Vectores}
  \begin{block}{Definición algebraica de un vector} \justifying
     Un \alert{vector $\bm{v}$} en el plano $xy$ es un par ordenado de números reales $(a, b)$. Los
números $a$ y $b$ se denominan \alert{elementos} o \alert{componentes} del vector $\bm{v}$. El \alert{vector cero} es el vector $(0, 0)$.
\end{block}

\lstinputlisting[firstline=10, lastline=23]{scriptsLinAlg/11_plot-vectors.py}
\end{frame}

\begin{frame}{Combinaciones Lineales}
  \begin{block}<only@1>{Combinación Lineal} \justifying
    Sean $\bm{v}_1, \bm{v}_2, \ldots, \bm{v}_n$ vectores en un espacio vectorial $V$. Entonces cualquier vector de la forma \[a_1v_1+ a_2v_2+ \cdots+ a_nv_n \] donde, $a_1, a_2, \ldots, a_n$ son escalares se denomina una \alert{combinación lineal} de $\bm{v}_1, \bm{v}_2, \ldots, \bm{v}_n$
  \end{block}

  \begin{onlyenv}<2>
      \begin{columns}[t]
    \column{0.5\textwidth}
    \lstinputlisting[firstline=3,lastline=14]{scriptsLinAlg/12_linear-combinations.py}
    \column{0.4\textwidth}
    \lstinputlisting[firstline=16]{scriptsLinAlg/12_linear-combinations.py}
  \end{columns}
  \end{onlyenv}
\end{frame}


\begin{frame}{Espacios y Subespacios}
  Los conjuntos $\mathbb{R}^2$ y $\mathbb{R}^3$ junto con las operaciones de suma de vectores y multiplicación por un escalar se denominan \alert{espacios vectoriales}. Se puede decir, de forma intuitiva, que un espacio vectorial es un conjunto de objetos con dos operaciones que obedecen las reglas que acaban de escribirse.
\end{frame}


\begin{frame}{Independencia Lineal}
  

      Sean $\bm{v}_1 , \bm{v}_2 , \ldots , \bm{v}_n$ , $n$ vectores en un espacio vectorial $V$. Entonces se dice que los vectores son \alert{linealmente dependientes} si existen $n$ escalares $c_1 , c_2 ,\ldots, c_n$ \emph{no todos cero} tales que $c_1 \bm{v}_1 + c_2 \bm{v}_2 + \cdots + c_n\bm{v}_n = \bm{0}$.  Si los vectores no son linealmente dependientes, se dice que son \alert{linealmente independientes}.


    ¿Existe alguna relación especial entre los vectores $\bm{v}_1 =
    \begin{pmatrix}
      1\\
      2\\
    \end{pmatrix}
$ y $\bm{v}_2 =
    \begin{pmatrix}
      2\\
      4\\
    \end{pmatrix}$? ¿Qué tienen de especial los vectores $\bm{v}_1 =
    \begin{pmatrix}
      1\\
      2\\
      3\\
    \end{pmatrix}
$, $\bm{v}_2 =
    \begin{pmatrix}
      -4\\
      1\\
5\\
    \end{pmatrix}$ y $\bm{v}_3 =
    \begin{pmatrix}
      -5\\
      8\\
19
    \end{pmatrix}$?
\end{frame}

\begin{frame}{¿Existe la Inversa de Una Matriz?}
\only<1>{%
  Una matriz es singular si tiene un vector linealmente dependiente en sus filas o columnas. Si removemos el vector linealmente independiente de la matriz, la matriz deja de ser cuadrada.
  
  \lstinputlisting[firstline=3]{scriptsLinAlg/15_singular.py}
}

\begin{block}<only@2>{ Valor característico y vector característico}\justifying
  Sea $A$ una matriz de $n \times n$ con componentes reales. El número $\lambda$ (real o complejo) se denomina \alert{valor característico} de $A$ si existe un vector diferente de cero $\bm{v}$ en $\mathbb{C}^n$ tal que \[\bm{Av} = \bm{\lambda v} \]

  El vector $\bm{v} \neq \bm{0}$ se denomina \alert{vector característico} de $A$ correspondiente al \alert{valor característico} de $\lambda$
\end{block}

\begin{exampleblock}<only@3>{Ejemplo}\justifying
  Sea $A =
  \begin{pmatrix}
    10 & -18 \\
    6 & -11 \\
  \end{pmatrix}
$. Entonces $A
\begin{pmatrix}
  2\\
  1\\
\end{pmatrix}
=
\begin{pmatrix}
    10 & -18 \\
    6 & -11 \\
  \end{pmatrix}
  \begin{pmatrix}
  2\\
  1\\
\end{pmatrix}
=
  \begin{pmatrix}
  2\\
  1\\
\end{pmatrix}
$. Así, $\lambda = 1$ es un valor característico de $A$ con el correspondiente vector característico $\bm{v_1} =
\begin{pmatrix}
  2\\
  1
\end{pmatrix}
$
\end{exampleblock}
\begin{exampleblock}<only@4>{Otro ejemplo de Vectores y Valores Propios} \justifying
  $A
\begin{pmatrix}
  3\\
  2\\
\end{pmatrix}
=
\begin{pmatrix}
    10 & -18 \\
    6 & -11 \\
  \end{pmatrix}
  \begin{pmatrix}
  3\\
  2\\
\end{pmatrix}
=
  \begin{pmatrix}
  -6\\
  -4
\end{pmatrix}
= -2
  \begin{pmatrix}
  3\\
  2
\end{pmatrix}
$, de modo que $\lambda_2 = -2$ es un valor característico (propio) de $A$ con el correspondiente vector característico (propio) $\bm{v}_2 =
\begin{pmatrix}
  3\\
  2
\end{pmatrix}
$.
\end{exampleblock}
\end{frame}


%%% Local Variables:
%%% mode: latex
%%% TeX-master: "../slides"
%%% End:
